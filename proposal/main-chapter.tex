%%% Preamble
%\title{Thesis Proposal}
\documentclass[paper=a4, fontsize=11pt]{scrartcl}
\usepackage[T1]{fontenc}
\usepackage{fourier}
\usepackage{comment}

\usepackage[english]{babel}
\usepackage[protrusion=true,expansion=true]{microtype}
\usepackage{amsmath,amsfonts,amsthm} % Math packages
\usepackage[pdftex]{graphicx}
\usepackage{url}
\usepackage{natbib}

%%% Custom sectioning
\usepackage{sectsty}
\allsectionsfont{\centering \normalfont\scshape}

%%% Custom headers/footers (fancyhdr package)
\usepackage{fancyhdr}
\pagestyle{fancyplain}
\fancyhead{}								% No page header
\fancyfoot[L]{}								% Empty
\fancyfoot[C]{}								% Empty
\fancyfoot[R]{\thepage}						% Pagenumbering
\renewcommand{\headrulewidth}{0pt}			% Remove header underlines
\renewcommand{\footrulewidth}{0pt}			% Remove footer underlines
\setlength{\headheight}{13.6pt}

%%% Equation and float numbering
\numberwithin{equation}{section}		% Equationnumbering: section.eq#
\numberwithin{figure}{section}			% Figurenumbering: section.fig#
\numberwithin{table}{section}			% Tablenumbering: section.tab#


%%% Maketitle metadata
\newcommand{\horrule}[1]{\rule{\linewidth}{#1}} 	% Horizontal rule
\newcommand{\quotes}[1]{``#1''}

\title{
		%\vspace{-1in}
		\usefont{OT1}{bch}{b}{n}
		\normalfont \normalsize \textsc{Chalmers University of Technology and University of Gothenburg} \\
        \normalfont \normalsize \textsc{Department of Computer Science and Engineering} \\
        \normalfont \normalsize \textsc{Software Engineering Master Thesis Proposal, 30hec
}\\[25pt]
		\horrule{0.5pt} \\[0.4cm]
		\huge Post-adoption of agile in large scale organisations: a case study investigating their interrelation \\
		\horrule{2pt} \\[0.5cm]
}
\author{
		\normalfont 				                     \normalsize
      Anna Averianova and Tobias Deekens\\[-3pt]	\normalsize
      \today
}
\date{}

%%% Begin document
\begin{document}
\maketitle

\begin{table}[h]
	\centering
    \begin{tabular}{ll}
    EDA397/DIT191  & Agile Development Processes             \\
    DAT250/DIT842  & Software Project and Quality Management \\
    DAT255/DIT599  & Software Evolution Project              \\
    DAT245/DIT278  & Empirical Software Engineering          \\
    \end{tabular}
    \caption {Completed courses relevant for thesis work (Chalmers/GU)}
\end{table}

\section{Introduction}

Agile Software Development's ideas and principles go back to the early 1960s and have been laid down in 2001 in the Agile Manifesto~\citep{beck2001agile}. Ever since the industry's need to adjust and respond to change quicker to deliver higher business value did not lose importance. Partially promising these results, agile is geared towards being applied within small, loosely coupled teams all working mostly independently~\citep{stober2009agile}. Embracing and transitioning towards any new work environment embodies barriers on a people and organizational level, both in their scale of magnitude being company dependent~\citep{schiel2009enterprise}. The general acceptance of Agile grows towards 84\% but it is mostly applied within companies of intermediate size~\citep{7thagilesur}. This is where bigger corporations tend to be confronted with a larger set of issues brought about by more defined practices and processes and a well-defined organizational structure~\citep{schiel2009enterprise}. These may not be static and subject to constant adjustments but can not be fully left behind in favor of agile as they are motivated by a need for coordination within a big scale.

\section{Statement of the problem}

Research has largely focused on general advice towards transformations from waterfall to agile development models, characterization and assessment of process maturity and the use of comparative measurements for its assessment. Nevertheless, it has less been concerned with belated integration difficulties that follow after the adoption of agile methods~\citep{ivari2011orgagile}. Still, this area is worthy of attention as agile's popularity increases.

The study will be conducted in cooperation with Ericsson's department at Lindholmen, PDU LMR/Configuration and O\&M. Dissatisfied with performance, around 2008 Ericsson started a transformation from the waterfall based development towards a more agile approach, following small incremental and discontinuous transformation steps. Rather than only changing the lower level coordination of development teams, it was decided to change the organizational structure along the way. A matrix-like organizational structure was replaced with hierarchical one with cross-functional teams at its lower level trying to embrace agile software development: a structure not necessarily prescribed by agile but motivated by Ericsson's scale.
The strictly hierarchical structure causes a great number of connections, clear responsibilities and therefore delegation and potential queuing delays. As performance fluctuations and discrepancies became visible over the course of the transformation, it seems that especially boundaries and connection within the structure may hinder taking full advantage of agile software development.

The thesis' aim is to investigate and question agile's values compatibility with large organizations' structures and cultures. This includes problematic aspects with agile at scale and possible flow and blockages, with a special focus on communication paths and their intersections. Investigations will focus on working out strengths and weaknesses within the interplay of agile's adoption and its coordination within the bigger picture, ultimately to move towards fruitful environments fostering self-management and productivity. For this purpose Ericssons' software department at Lindholmen cooperates and will provide valuable insights regarding the thesis' research objectives.

\section{Purpose of the study}

Research has largely focused on general advice towards transformations from waterfall to agile development models, characterization and assessment of process maturity and the use of comparative measurements for its assessment. Nevertheless, it has less been concerned with belated integration difficulties that follow after the adoption of agile methods~\citep{ivari2011orgagile}. Still, this area is worthy of attention as agile's popularity increases.

The study will be conducted in cooperation with Ericsson's department at Lindholmen, PDU LMR/Configuration and O\&M. Dissatisfied with performance, around 2008 Ericsson started a transformation from the waterfall based development towards a more agile approach, following small incremental and discontinuous transformation steps. Rather than only changing the lower level coordination of development teams, it was decided to change the organizational structure along the way. A matrix-like organizational structure was replaced with hierarchical one with cross-functional teams at its lower level trying to embrace agile software development: a structure not necessarily prescribed by agile but motivated by Ericsson's scale.
The strictly hierarchical structure causes a great number of connections, clear responsibilities and therefore delegation and potential queuing delays. As performance fluctuations and discrepancies became visible over the course of the transformation, it seems that especially boundaries and connection within the structure may hinder taking full advantage of agile software development.

The thesis' aim is to investigate and question agile's values compatibility with large organizations' structures and cultures. This includes problematic aspects with agile at scale and possible flow and blockages, with a special focus on communication paths and their intersections. Investigations will focus on working out strengths and weaknesses within the interplay of agile's adoption and its coordination within the bigger picture, ultimately to move towards fruitful environments fostering self-management and productivity. For this purpose Ericssons' software department at Lindholmen cooperates and will provide valuable insights regarding the thesis' research objectives.

\section{Review of the literature}

Agile promises to yield several benefits such as reduced time-to-market, increased quality and the ability to respond to market change quicker, thus it is becoming a competitive advantage to apply agile correctly~\citep{schwaber2007agile}. Adopting some of agile's methodologies is not a silver bullet and a straight route to success. As pointed out by~\citet{grenning2001xp}, the utilization of eXtreme Programming (XP) embodied unexpected issues mostly caused by varying personal expectations especially on different levels of the organization. The importance of understanding the organization's status quo first in order to improve towards becoming more agile has already been emphasized early by~\citet{zhang1999manufacturing} for manufacturing businesses.~\citet{kettunen2008agileorg} take this approach further stressing the fact that some parts of an organization are more leaning towards agile than others. The same study also identified clear friction between larger corporations' business models and an unpredictable development process. By the same token agile feedback loop tends to slow down by defined processes, complex dependencies and product life-cycles.

The above hints at a possible incompatibility between agile, organisational structures and expectations on various managerial levels. Nevertheless, an existing environment might entail inadequate measurement techniques towards agile. Metrics which as a quantifiable description of the development process have been widely discussed with respect to plan driven software development. Quite the contrary to an agile development process whose evaluation tends to differ significantly from its plan driven counterpart. As suggested by~\citet{hartmann2006appagilemes}, agile measurements should be tailored in accordance with the development process rather than simply adopted from plan-driven approach.~\citet{leffingwell2007scalelargecorps} on a related note proposes to evaluate a team among six axes from development practices to planning. All in all, the high degree of context sensitivity makes a universal application of metrics one can apply to deeply estimate the performance in most cases unfeasible. However, frameworks - such GQM (Goal-Question-Metric)~\citep{victor1994gqm} or Balanced Scorecard~\citep{kaplan1991balancesc} - exist to rationally choose the metric and connect it to a specific goal. Moreover, several attempts have been made to map agile software development to the CMM/CMMI model. As shown by~\citet{pitterman2000maturity} for example, reaching higher levels of the model assists in producing quality software while increasing productivity and saving budget. Both the possibility of such a mapping~\citep{paulk2011xpcmm} and conflicting areas between the two~\citep{turner2002agilecmmi} have been pointed out. Still, no definite agreement about the issue and a resolution has been agreed upon. Thus newly created maturity models, applicable in the agile context, have been suggested, such as AMM (Agile Maturity Model)~\citep{chetankumar2009amm} and LMM (Light Maturity Models)~\citep{buglione2011lmm}.

Moreover, culture is also often perceived with a predefined and static degree of fitness in terms of its agility~\citep{ivari2011orgagile}. Which is where academia has followed transformations from plan-driven to agile software development in organizations of various size to investigate possible obstacles and encountered issues~\citep{laanti2011nokia}. Here a lack of profound body of knowledge in the area of agile adoption in large organizations has been pointed out especially as new practices do not tend to be equally appreciated even among employees within a single level of the organisation. In this context ~\citet{ivari2011orgagile} abstract further and apply the Competing Values Model of organizational culture to propose a number of hypothesis to mostly relate hierarchical structures with agility. Anyhow, open ends remain in relation to an emerging perspective towards culture in which practices and beliefs are connected and everchanging. In this context,~\citet{gallivan2005persculture} pays significance to the fact that organizational structures and cultures have mostly been analysed separately and calls for an integrated analysis. In particular as company structures tend to be fixed in theory in contrast to their behavior in a working environment in which participants just be perceived more individual and with a set of layered forces~\citep{gallivan2005persculture}. Individuals are constantly reshaping and reinterpreting a given organisational structure to make it more compatible to a changing internal and external environment~\citep{iivari2010usability}. This belief is shared by~\citet{robey2000learning} who points out that organisational learning as a constant persistence of learnings in an evolving environment is usually underestimated in industry.

\section{Research questions}

An attempt answering the following research questions will be made.

\begin{enumerate}
   \item In which ways can an organizational structure hinder or foster information flow within large scale agile development?
   \item How can problematic aspects of work environments within cross-functional teams be identified and compared?
   \item To what extent can incompatibility between agile development and larger companies' organizational structures be identified and mitigated?
\end{enumerate}

\section{The Design \textemdash Methods and Procedures}

The \quotes{Statement of the Problem} will be investigated using a case study as the main research method. A representative subset of cross-functional teams will serve as the study's subject and their activities performed within the department itself as unit of analysis~\citep{thomas2011casestudy}. Identifying a relevant scope is crucial for the research's success accounted by the fact of Ericsson being one of the largest organizations having adopted agile methodologies. Thus sampling should be aligned to a diversity in subjects to help discovering a high degree of variance in collected information rather than saturation of a specific finding~\citep{vandeven2007socialresearch}.

With the case study's main purpose being exploratory, illustrative and retrospective, the existing organizational structure and interconnections between separate units will first be thoroughly investigated with the help of relevant documentation available on site in addition to interviews with assigned employees~\citep{thomas2011casestudy}. The currently existing development processes will be explored in a similar fashion which will also be complemented with observations. The case study's setting will thereby be uncontrolled regarding its influences and environmental variables~\citep{yin2009casestudy}.
Possible issues that may be causing the problems, which are the main objective of this study, will firstly be gathered from existing literature to later prove their validity using the given context. Issues other than those mentioned in the existing research will be obtained using qualitative data from the employees during interviews. The initial literature review's second intent is to guide through defining a thematic priority concerning the interviews. Gathered information in form of notes at any stage must be treated confidentially but exchanged and discussed between supervisors both at university and the company~\citep{yin2009casestudy}. The currently envisioned process will try to ground the problem in reality paired with a literature review and move from the study of identified problems towards their evaluation and assessment. This finally leads to causal relations between observations within the thesis' research questions and their possible origins being made.

Lastly, the study will look at the available historical data about performance (more classical metrics on software deliverables and check-ins) that have been continuously collected. These can be used to underpin qualitative aspects from a quantitative perspective, especially in areas in which time-oriented data can reveal delays.

\section{Limitations and Delimitations}

The study will be conducted in collaboration with a single organization hence the setting might be biased by the culture and structure of this particular organization and consequently it's interpretation of agile. The results therefore may not be generalizable to full extent, which is a threat to external validity of the study.

The literature review performed to select an initial set of possible causal factors might not be complete to fully explain the existing context. The study will attempt to discover other causes but some of them might go unnoticed by thesis's conductors, posing a threat to internal validity. Related to this is the bias of making the final selection of possible causes - a most likely highly subjective task. Unanticipated events, structural or other changes affecting the environment under study might have an influence on the workflow within the company and therefore interfere with the results. Greater parts of the study will be done using qualitative methods (namely, interviews), here the strategy for sampling of interviewees as well as the process of their conduct might also threaten the internal validity. The employees also might be subject to Hawthorne effect making the qualitative data not fully realistic. Lastly, the quantitative data available for analysis from the company may not be sufficient to describe the process completely from relevant perspectives and therefore might not guarantee reliable conclusions.

Misinterpretation of any aspect of the work process under study is a threat to construct validity. The discrepancies in used constructs might appear during the interviews, as well as between the studied research literature and company's internal documentation.

Only one case study will be performed during the thesis due to a limited amount of available resources. Interviews will be conducted instead of surveys as a consequence of high cost of the latter and as a trade-off towards a more in-depth case study.
Further studying the existing maturity models in the context of agile development process will not be performed. Their application will be left out from the thesis, as it will focus on the organizational structure. No new measurements will be introduced and collected due to a limited timeframe of the study not allowing to observe any reliable trends. Performance of cross-functional teams will not be assessed and literature related to this subject will not be reviewed as it was not identified as a problematic area of the development process by Ericsson PDU LMR/Configuration and O\&M.

\section{Significance of the study}

One of the main purposes of this study is to provide real data about the effects the organizational structure has on the agile software development process in the context of the large scale companies. 
Agile scalability is a sensitive topic within the industry as even larger companies have been adopting agile methodologies. This thesis will attempt to identify possible shortcomings of agile at scale with respect to organizational structure and thus provide the practitioners with a reference to potential areas of improvement in the context of their organizations. In addition, as mentioned by~\citet{ding2013research}, research in the area of large scale agile has not been extensive and \quotes{practice is ahead of research}. Hence, this thesis will attempt to complement the existing knowledge about difficulties accompanying agile integration in large scale enterprises.

\pagebreak
\bibliography{biblio}
\bibliographystyle{agsm}

%%% End document
\end{document}