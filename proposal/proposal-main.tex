\documentclass[times, 10pt,twocolumn]{article}
	\usepackage{proposal-style}
	\usepackage{times}
	\usepackage{comment}

	\pagestyle{empty}

	\begin{document}

	\title{A insanely sophisticated sounding title to be come up with.}

	\author{Anna Averianova\\
		Tobias Deekens\\
		Chalmers | University of Gothenburg\\
		Department of Computer Science and Engineering\\
	}

	\maketitle
	\thispagestyle{empty}

	\begin{comment}
		Agile Software Development rooted in the need to adjust and respond to change quicker to deliver higher business value. Most Agile development processes are geared towards being applied within small, loosely coupled teams all working mostly independent. The promises promoted by Agile’s advocates remain appealing to large scale organizations such as Ericsson. Yet not all benefits might be put into effect possibly due to an intertwined organizational structure.
	   Ericsson moved from a matrix-like organisational structure towards a hierarchical one with Cross-Functional-Teams at its lower level trying to embrace Agile Software Development: a structure not necessarily prescribed by Agile but motivated by Ericsson’s scale. As performance fluctuations and discrepancies became visible over the course of the transformation, it seems boundaries and connection within the structure may hinder taking full advantage of Agile Software Development.
	   A thesis’ aim is to investigate possible issues with Agile at scale, flow and blockages within Ericsson’s organizational structure and the creation of fruitful environments fostering self-management and productivity. A more concrete research scope will be determined while decisions upon methods will be made in accordance with the final scope. These could include interviewing, surveying, observations, access to software repositories or other existing relevant data.
	\end{comment}

	\Section{Introduction}

	\begin{comment}
	  - Cover a bit of it all without crazy detail... (listed in template too)
	  - Leave Ericsson mostly out for now, they're just the subject to investigation not main objective
	  		- Do not mention research design yet
	  - Introduce field of research (just a line or two): Agile its roots and industry acceptance
	  - Reasoned by hitch of small scale development methodologies in large scale corporations
	  - Research context by mentioning Agile at scale with transformation by big firms
	  - Narrow the scope towards organizational structures
	\end{comment}

	% Enter the blabla - its just the 1st shot at it

	Agile Software Development’s ideas and principles go back to the early 1960s and has been layed down in 2001 in the Agile Manifesto~\cite{beck2001agile}. Ever since the industry’s need to adjust and respond to change quicker to deliver higher business value did not loose importance. Agile partially promising these results, is geared towards being applied within small, loosely coupled teams all working mostly independently~\cite{stober2009agile}. Embracing and transitioning towards an environment embodies barriers on a people and organizational level, both in their scale of magnitude being company dependend~\cite{schiel2009enterprise}. The general acceptance of Agile grows towards 84\% but is mostly applied within companies of intermediate size~\cite{7thagilesur}. This is where bigger corporations tend to be confronted with a larger set of issues brought about by more defined practices and processes and a well-defined organizational structure~\cite{schiel2009enterprise}. Organizational structures which may also be subject to further transformations but will mostly not be fully left behind as they are motivated by a need for coordination. Still, an organizational structure may be more or less fitting and have an need for optimizations.

	% ... needs some proof I suppose
	Research has mainly been focussing on general advice towards transformations, maturity and comparative measurements and has less been concerned with belated integration difficulties. A thesis’ aim is to investigate possible issues with Agile at scale in regard to information flow and blockages within organisational structures.

	\Section{Statement of the problem}

	\begin{comment}
	  - Is this to be company specific, only research fueled or a combination of the two?
	  - Having slight problems wrapping my head around disentangling problem and literature review...
	\end{comment}

	\Section{Purpose of the study}

	The purpose of this study is to help Ericsson gain world dominance by applying Agile methodologies at scale while allowing the researchers to finish with a knightly accolade.

	\Section{Review of the literature}

	\begin{comment}
	  - Seperate ourselves from
	  		- Traditional metrics: cost, value etc
	  		- Traditional maturity models: focussing on application of methogologies mainly
	  - Move towards a disregard to Agile with organizational structures in literature
	  		- Integration, information flow etc... link back to purpose
	  - Mention Agile at Scale to emphasize issues
	\end{comment}

	\Section{Research question and/or Hypotheses}

	Lorem ipsum.

	\Section{The Design – Methods and Procedures}

	Lorem ipsum.

	\Section{Significance of the study}

	Lorem ipsum.

	\bibliographystyle{proposal-biblio-style}
	\bibliography{biblio}

\end{document}