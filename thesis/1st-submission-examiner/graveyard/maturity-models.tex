
\todo{I think this subsection has to go.}
\section{Agile Maturity Models}

The above hints at a possible incompatibility between agile, organisational structures and expectations on various managerial levels. Metrics which as a quantifiable description of the development process have been widely discussed with respect to plan driven software development. Quite the contrary, an agile development process' evaluation tends to differ significantly from its plan driven counterpart. As suggested by~\citet{hartmann2006appagilemes}, agile measurements should be tailored in accordance with the development process rather than simply adopted from plan-driven approach.
Thus, several attempts have been made to map agile software development to the~\ac{CMM}/\ac{CMMI}. As shown by~\citet{pitterman2000maturity} for example, reaching higher levels of the model assists in producing quality software while increasing productivity and saving budget. 
Still, heavy weight process models are not applicable to most organizations. Here~\citet{chetankumar2009amm} emphasizes a mismatch between agile and lean software development processes and~\ac{CMMI} objectives and focus. Its overly complex nature tends to generate needlessly high cost while not focussing on the software development process itself. This lead to the creation of new maturity models, applicable for the agile context, such as~\ac{AMM}~\citep{chetankumar2009amm} and~\ac{LMM}~\citep{buglione2011lmm}.The ~\ac{AMM} proposes possible software process improvements on five levels from a business perspective based on agile's practices and values. 

Other attempts tried to define a mapping between~\ac{KPA}s and~\ac{XP}'s life cycle activities to pair a system perspective of programming with an alignment of organisational process improvements~\citep{paulk2011xpcmm}. General support of~\ac{MM}s is pointed out in~\citet{turner2002agilecmmi} findings stating that gaps between process improvement and agile are gaining more understanding. An insight which should result in hybrid approaches incorporating ideas from both worlds dependent on the application's context.
Consequently, the~\ac{LMM} takes an even more simplified approach aiming to bridge the gap between the application of no maturity model and normal~\acs{MM}s. It links drivers to agile values to create a continuous path towards higher capabilities~\citep{buglione2011lmm}.
~\citet{leffingwell2007scalelargecorps} on the other hand suggests to perform Agile Process Assessment Metrics on a team level in which principles and objectives are ranked to visualise results among six axes relating to their nature. The evaluation of an organisation as a whole should then be conducted in an approach using~\ac{BSCs} with categories concerned with efficiency, value delivery, quality and agility. Hereby, each unit within the organisation could be ranked and depicted again in overall structure accordingly indicating its score in each field~\citep{leffingwell2007scalelargecorps}.

Justified by the diversity in~\acs{MM}s,~\citet{2013schweigertamm} aimed at aggregating current similarities by developing a reference model. All models' levels are related to~\ac{CMMI}'s counterparts pointing out inconsistent naming. They conclude that~\acs{MM}s should not be seen on a two dimensional scale but rather as a spider web. Automatically extracting characteristics from 12 models and a later manual mapping process helped finding groups of concerns. Hereby axes for the spider web were defined in Process Capability, Technical quality, Organizational support for agile development, Agile teaming and Agile culture. 

All in all, the high degree of context sensitivity makes a universal application of metrics one can apply to deeply estimate the performance unfeasible in most cases. This is why frameworks, such~\ac{GQM}~\citep{victor1994gqm} or the mentioned~\ac{BSCs}~\citep{kaplan1991balancesc}, exist to rationally choose the metric and connect it to a specific goal or setting up the basis for it~\citep{2013schweigertamm}. 