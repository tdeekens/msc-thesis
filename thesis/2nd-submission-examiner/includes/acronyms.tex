\chapter*{Acronyms \& Abbreviations}

\begin{acronym}

\acro{APO}{Area Product Owner}
\acro{DM}{Department Manager}
\acro{DSD}{Distributed Software Development}
\acro{FPjM}{Feature Project Manager}
\acro{OPO}{Operative Product Owner}
\acro{PDU}{Product Development Unit}
\acro{PDU LMR}{Product Development Unit Long Term Evolution and Multistandard Radio Base Stations}
\acro{PG}{Product Guardian}
\acro{PgM}{Program Manager}
\acro{PO}{Product Owner}
\acro{SM}{Section Manager}
\acro{STC}{Socio-Technical Congruence}
\acro{TC}{Team Coach}
\acro{TPO}{Total Product Owner}
%\acro{CMMI}{Capability Maturity Model Integration}
%\acro{CMM}{Capability Maturity Model}
%\acro{CVM}{Competing Values Model}
%\acro{GQM}{Goal-Question-Metric}
\acro{XFT}{Cross-Functional Team}
\acro{XP}{eXtreme Programming}

\end{acronym}

\begin{comment}

- \ac{abbr}   - standard, enters abbr but fully explained on first mention
- \acs{abbr}  - always only abbr
- \acf{abbr.} - always abbr and full explanation
- \acl{abbr}  - only explanation

\end{comment}