\chapter{Related Work}
\label{chap:related-work}

Prevailing research of interest for this study mainly spans across areas which cover application of agile methodologies at a large scale. First, a foundation is laid by outlining expectations and implications of agile's application which then leads to more profound explanations regarding agile at scale and the influence of agile on communication and coordination to lastly give an outline of related work in the field of~\ac{DSD}.

\section{Expectations and Implications of Applying Agile}

It is important to note that agile left alone with its mentioned core principles and values does not aim at increased productivity or alignment with business expectations. The focus is upon agility itself to move fast without imposed friction embracing change while overcoming its obstacles~\citep{larman2008scalingleanagile}. In that regard, the Agile manifesto has been recently scrutinised as its values are perceived as too vague for either academic research, business application or methodology development~\citep{laanti2013definitionsofagile}.
Nevertheless, the principles have frequently been used as input to set up new methods attempting to yield business benefits by increasing productivity, quality and satisfaction~\citep{leffingwell2007scalelargecorps}. It has been found that developer satisfaction increases with agile methodologies and especially \ac{XP} in place, also resulting in an organisation becoming more attractive and generally pleasant for employees to work at~\citep{eckstein2004xp}.~\citet{mann2005casestudyimpactscrum} add, that overtime tends to be decreased in agile projects by enabling developers to work faster and communicate workloads more appropriately towards customers.
With agile's success and recent applications with large scale in mind,~\citet{cockburn2005doi} defined the Declaration Of Independence (DOI) linking people, project and values to emphasize the interdependent nature of a network each project team resides in. A lack of awareness and care of this network will limit the likelihood for teams to succeed with their mission.

Agile promises to yield several benefits such as reduced time-to-market, increased quality and the ability to respond to market change quicker, thus it is becoming a competitive advantage to apply agile correctly~\citep{schwaber2007agile}. 
As pointed out by~\citet{cardozo2010scrumproductivity}, successful development tends to have a strong correlation with agile, customer satisfaction and team motivation. Moreover, agile positively affects different levels of organisations as stated by \citet{ceschi2005projman}, who discovered that project management in general benefits by tighter customer collaboration and the ability to embrace changing requirements more easily.
Still, adopting some of agile's methodologies is not a silver bullet and a straight route to success. As pointed out by~\citet{grenning2001xp}, the utilization of~\ac{XP} embodies unexpected issues mostly caused by varying personal expectations especially on different levels of the organisation. The importance of understanding the organisation's status quo in order to improve towards a higher level of agility has been emphasized.~\citet{kettunen2008agileorg} stress the fact that some parts of an organisation are more leaning towards agile than others. Furthermore, different parts of an organisation might be generally favouring a change trying to eagerly deploy it while other parts may evoke opposition \citep{cohn2003introducingagiletoanorganisation}.~\citet{kettunen2008agileorg} also identify a clear friction between larger corporations' business models and an unpredictable development process. By the same token, the agile feedback loop tends to be slown down by defined processes, complex dependencies and product life-cycles.

Taken together, the introduction of agile to previously existing organisations bears threats to its successful implementation often also relating to the context's scale.

\section{Agile at Scale}

Agile's growing support and appreciation leads to it being used in large contexts or even throughout multi-site organisations.~\citet{larman2008scalingleanagile} emphasize, that Scrum should not be evolved towards a new general methodology for a scaled context. It should rather remain a set of roles and ideas which every organisation takes into consideration and adjusts to their needs. A loose definition may leave ground for misunderstandings giving little advice for larger organisations, but has advantages as it allows for a custom application~\citep{laanti2013definitionsofagile}. 
Spotify \citep{kniberg2012agilespotify}, for instance, embraces the notion of flexible guidelines within agile to scale it differently from suggested approaches by, for example, \citet{larman2008scalingleanagile}. Scaling in itself should be aligned to the product and its expectations and leave room for design of beneficial communication flows and coordination between teams, as it mostly evolves among the multiple teams or even units~\citep{larman2008scalingleanagile}. In any case, dependencies between teams increase with the number of concurrently working teams, in turn causing constraints and blockages on the order of tasks carried out~\citep{moore2008scalingagile}.
In this regard~\citet{nelson2013technicaldependencies} investigate the challenges associated with technical dependencies and point out that these impinge each other, causing domino effects and potentially vicious circles thus blocking progress.

The adoption of agile practices within an established enterprise environment also entails issues extending solutions prescribed by the standard perception of agile at scale~\citep{larman2008scalingleanagile}. Moving towards a new process and leaving old, established ways of working behind brings change which is not necessarily appraised equally among the organisation~\citep{benefield2008agileenterprise}. Here especially educating practices and continuous coaching are perceived as vital to success to lower dysfunctional patterns, such as managerial control over teams and resources, as high level business procedures can not always be changed and have a proven right to exist.~\citet{turk2000limitationsagile} question agile's applicability for any context and suggest classifying processes among a spectrum of agility allowing a more context sensitive application, reasoned in the fact that any distinct mix of methods yields different outcomes for the overall development process.~\citet{eklund2012agileembedded} focus on  embedded systems development and propose a model for applying agile within a plan-driven corporate environment. Their investigations put significance on interactions between the agile and plan-driven parts of the organisations. They argue that the awareness and definition of interactions regarding requirements, product project gates and validation reduces friction with non agile parts of the organisation. The border should be wisely placed as the application of Scrum solely on a team level excludes synchronisation on a product level and among development teams~\citep{laanti2008implprogrammodel}. By the same token, management concerns and long-term planning should be combined with short-term development strategies leading to longer development cycles but improved alignment with product visions.

~\citet{boehm2005challengesagile} take a broader view and categorise the potential problems to be mostly of three natures: development-, business process or people conflicts. Each of these have different root causes and means to be addressed. Mitigating and avoiding software variabilities and evolving legacy systems can be achieved by thorough planning, risk awareness and creating a customized process. Such methods, according to~\citet{boehm2005challengesagile}, do not necessarily align with agile's principles but are complementary to piloting projects and continuous measurements to react to business process challenges, which also embody expectations carried over from previous processes.

\citet{kettunen2008agileorg} continue from accepting the heterogeneous distribution of agility within an organisation by extending a model to systematically assess agility of separate units. At its current state it envisions understanding and measuring agility through understanding enablers, means and goals. Gained understanding might also reveal local optimizations and unintended accumulations of knowledge.

All in all, the application of agile within a large-scale organisation bears the potential of yielding its promised improvements. The flexibility towards its application eases the adoption, but also puts forward challenges on various levels of an organisation. 

\section{Agile, Coordination and Communication}

Incorporating and scaling methodologies within an existing organisational culture, which comprise a redefinition of established ways of working, will affect communication and coordination.

Agile software development itself relies heavy on internal communication within a team and external communication with the customer. It embraces the high degree of tacit knowledge aiming at reducing the need for formal documentation~\citep{beck2001agile}. A high degree of informal communication within~\acs{XFT}s however does not inevitably ensure a project's success. A potential lack of communication between different roles which do not directly fit into agile practices can entail threats to success~\citep{coram2005impactagileprojman}. Especially as uncertainty about ways of external communication from and towards development teams comprises potential to optimise collaboration outcomes~\citep{svensson2005viewsagilecollaboartion}.

\citet{pikkarainen2008impactagilecommunication} point out that agile methodologies increase formal and informal communication within an organisation as intended and anticipated by practitioners. Nevertheless, ways of handling the increasing amount of available information are not always in place limiting its potential advantages. For instance, information regarding long-term goals of multiple dependent features can be present in some but not communicated to all parts within the organisation \citep{pikkarainen2008impactagilecommunication}. \citet{cohn2003introducingagiletoanorganisation} argue that even though agile increases communication within development units, upper management quickly loses the ability to track progress and to plan and control the underlying development process. This loss of control is to be expected by structural empowerment 
which aims at distributing and delegating the decision making power towards development units~\citep{millslimitsempowerment}. According to~\citet{tessemindividualempoqerment}, agile developers are often more empowered by gaining a higher level of managerial influence and task selection possibilities leading to an increased work motivation.
Still, the empowerment and shift of responsibilities often causes friction between parts of an organisation caused by scepticism and uncertainty eventually resulting in counter-productive behaviour~\citep{millslimitsempowerment}. 

The coordination within the development of large software systems has become one of the main managerial challenges and shows limits for empowerment \citep{kraut1995coordinationinsd}. Large-scale systems of high complexity entail a high level of interdependence of separate components developed by a large number of teams. According to \citet{kraut1995coordinationinsd}, the coordination mostly relies on informal communication which does not solve issues around the search for a consensus and information sharing. Coordination grows to become particularly challenging in software development with the rise of~\ac{DSD}. Distance tends to hamper communication which is the main intermediate towards engaging in collaboration and control with projects. With the great chance of communication being too sparse or single pieces being distorted, threats for the whole software development process, such as activities within requirements engineering, arise \citep{prikladnicki2003globalsoftware}.       

In general, areas of concern regarding coordination and communication in respect to agile tend to be manifold and span over an organisation as a whole and can be seen as an enduring challenge not to be overcome at a specific moment in time.

\section{Communication in~\acl{DSD}}

Communication and coordination issues do not solely become apparent in large-scale agile projects and also have been of interest within~\ac{DSD}. The degree of distribution in software development, according to~\citet{cockburn2007agile}, can be ranked on a scale where local co-location corresponds to the lowest values while global distribution is of highest degree of distribution.

\citet{curtis1988fieldstudysoftwaredesign} state that a scarcity of informal communication channels negatively impacts software development in general. This problematic aspect gains even more importance as software components' growing size and complexity strictly correlates with a higher demand of informal communication for coordination~\citep{kraut1995coordinationinsd}. At the same time~\citet{kraut1995coordinationinsd} acknowledge the existence of communication barriers which root in organisational, social or geographical differences and all have the possibility to restrict the ability or will of units within the organisation to engage in communication and share information. As the frequency of communication generally drops with increasing distribution of development,~\citet{herbsleb2003empiricalcommunication} discover a restricted flow of information caused by little interaction. This ultimately leads to employees feeling distant and less knowledgeable about the overall direction and plan of the organisation~\citep{herbsleb2003empiricalcommunication}. The channels and ways to address the decline of informal communication have been subject to studies with different impacts on the development's success~\citep{niinimaeki2008im, smite2006comm}.~\citet{rbs2012softwarearchitecture} evaluate different communication mechanisms categorised in three main groups claiming that some are more applicable than others in local and global site communication. Mechanisms differ according to richness, the ability to transfer volume and information on a timely manner. Global and local development in turn have different demands towards communication leading to different mechanisms being favoured and relied upon~\citep{rbs2012softwarearchitecture}. Architecture has been identified as the most applicable communication mechanism for developing reusable software components as it manages to minimise impacts of distance and cultural and language differences. Still, the importance of direct face-to-face communication is highlighted for development teams as a fast communication mechanism even when it fails at times to communicate large volumes of information~\citep{rbs2012softwarearchitecture}.

\ac{DSD} also spreads dependencies and increases the geographical distribution between software components. In this context~\citet{ovaska2003architecture} found a set of dependencies between components and development activities such as information diffusing between work activities not being understandable by all parties. This is mainly caused by different degrees of knowledge of a piece of information for the receiver and sender where both parties are unable to find common ground and communicate their concern adequately. Furthermore,~\citet{ovaska2003architecture} states that by responsibilities spreading with the distribution, a lack of a overall hierarchy impacts the decision making ability within development. It shall be noted that the mentioned issues do not solely arise from a geographical distribution but also arise within single-site organisations~\citep{ovaska2003architecture}. 

The alignment of the actual amount of communication and its relation to an expected coordination need is referred to as~\ac{STC}~\citep{cataldo2006identcoord}. A higher level of~\ac{STC} is usually attributed to increase development productivity and improve team coordination~\citep{herbsleb2007gse}.~\citet{damian2013domainknowledge} analyse~\ac{STC} around requirement-based dependencies pointing out that team members often communicate with others having similar knowledge and domain experts tend to act as communication hubs spanning wider over a social network than normal actors. This relates to Conway's law which states that organisations, such as software development firms, create designs which are heavily influenced or even constrained by their communication structures~\citep{conway1968}. This view is supported by~\citet{coplien2004orgstruc}, who heavily highlight the need for an equal relationship between product parts and the organisational units to avoid later integration problems.

All in all, communication has been proven to be one of the main and costly challenges in~\ac{DSD} whose alleviation potentially bears productivity improvements for development.