\chapter{Case Company Description}
\label{chap:case-company-description}

This chapter introduces the company under study by commenting on its agile transformation, the organisational structure, and associated roles of study participants altogether providing the description of a large-scale agile application peculiarities.

\section{History of the Transformation}

Increasing productivity remains the main purpose towards the application of agile methodologies in industry. Nevertheless,~\citet{badampudi2013proddelay} point out that most companies adapting agile do not manage to strictly follow all of its main ideas. Adjustments are made to integrate agile with the large scale and existing processes, with both positive and negative impacts upon productivity. Similarly, Ericsson has its own peculiarities of scaling agile which will be discussed in this chapter.

Dissatisfied with performance, several years ago the~\ac{PDU LMR} organisation at Ericsson started a transformation from the waterfall based development towards a more agile approach, following small incremental and discontinuous transformation steps. Rather than only changing the lower level coordination of development teams, it has been decided to change the organisational structure along the way. A matrix-like organisational structure was replaced with hierarchical one with~\acp{XFT} at its top trying to embrace agile software development: a structure not necessarily prescribed by agile but motivated by Ericsson's scale. The strictly hierarchical structure causes a lower number of connections, clear responsibilities and delegation but embodies potential queuing delays. 

This is justified by the fact that matrix structures in general and as previously employed by Ericsson, combine functional (divided by types of work) and divisional (product based division) structures, adding another horizontal line of communication~\citep{price2007hrm}. Hence, each unit within the structure is being coordinated by two superior entities: a functional- and a divisional superiority. The intent is to faster distribute knowledge horizontally among functional sectors without having to move through a long chain of hierarchies~\citep{galbraith2008matrix}. 
Hierarchical structures on the other hand are solely an extension to functional or divisional structures adding a chain of command and show superior and subordinate units or roles. The hierarchy does not imply horizontal communication and becomes narrower towards its top~\citep{healeyprojectmanagement}.

\ac{PDU LMR} integrated parts of agile's methodologies into their hierarchical organisational structure while adding variations where needed. This includes partially deploying methodologies, defining custom roles and responsibilities, and adding an integration layer for the organisational structure.

\section{Organisational Structure at Ericsson~/ac{PDU LMR}}
\label{Chap:OrgStrucEricsson}

The structure of the organisation under study is presented on the Figure \ref{fig:org-structure}.

\begin{figure}[h!]
  \centering
  \includegraphics[width=0.90\textwidth]{figures/organisational-structure.pdf}
  \caption{Organisational structure of~\ac{PDU LMR}}
  \label{fig:org-structure}
\end{figure}

Roles with a strict relation to agile methodologies are illustrated with an ellipse and more thorough description of their responsibilities can be found in Table 3.1. Roles are grouped horizontally and the connecting lines outline the organisation's hierarchy and intended chain of commands.

In the epicentre of the development activities reside multiple~\acp{XFT} — self-sufficient units which have all the necessary competencies for feature delivery at their disposal.~\acp{XFT} generally consist of five to nine members. 
\acp{PG} are not part of an~\ac{XFT} but are also assigned to different products in order to oversee development and maintain a product's coherence.~\acp{XFT} working on parts of a whole product in turn interact with one or several~\acp{PG} which for teams also eases handling of its size and complexity.

Each team belongs to a section with a~\ac{SM} looking over two or more~\acp{XFT}. The~\ac{SM} also acts as a~\ac{TC}, whose responsibilities are described later in the Table 3.1.
Sections themselves belong to departments, which in turn are a part of a sector, each of them having a respective manager. A structure above the sector in the hierarchy is called a~\ac{PDU}. Together, this part of the structure constitutes the line organisation. The line is further supervised by a Design Unit (DU) and a Business Unit (BU) which are not part of further discussions.

The start of the transformation towards an agile development process caused the addition of a new product owner community to the existing structure (depicted on the right hand side of Figure~\ref{fig:org-structure}). Given the company's scale, the traditional role of a~\ac{PO} had to be divided into the areas of responsibility. Thus, new roles of~\acp{TPO},~\acp{APO} and~\acp{OPO} were introduced with a~\ac{TPO} being in direct contact with the customer and~\acp{APO}, who in turn work closely together with several~\acp{OPO} each.~\acp{OPO} themselves work with several~\acp{XFT} at a time. The exact amount depends on the nature of the product and a way of working inside a section. In case of a feature assigned to an \ac{XFT} being too large and complex, a~\ac{FPjM} acts as an intermediary between the~\ac{PgM} and~\acp{OPO}, where the former is responsible for maintaining the high-level backlogs teams eventually get the stories from.

\begin{figure}[h!]
  \centering
  \includegraphics[width=0.8\textwidth]{figures/onion.pdf}
  \caption{Layers of roles and their interaction}
  \label{onion}
\end{figure}

Figure~\ref{onion} illustrates the distance and an intended amount of collaboration between the roles by outlining layers of interaction. An~\ac{XFT} is always in the closest and immediate contact and cooperation with the~\ac{PG} of the product they work on, their~\ac{OPO} and~\ac{SM}. The next layer is comprised of those roles who have a frequent contact to the~\ac{XFT} and their environment while the degree of this contact is significantly lower than in the first layer. Hence, the bigger distance from the~\ac{XFT} to another role, the less communication is envisioned.

In the scope of the study two \acp{XFT} were investigated, where both teams have six members (including a Scrum Master) and work closely with their \ac{OPO} while having a different degree of contact with their respective \ac{PgM}. At the time of the study both \acp{XFT} did not have a \ac{SM} and his responsibilities were temporarily taken over by respective \acp{DM}. Only one team had a dedicated \ac{PG} while the responsibilities of this role were spread out between different persons for the other team. Summarised, these roles are referred to as \emph{immediate environment} of an \ac{XFT} further in the thesis.

\section{Role Descriptions \& Definitions}

The more detailed description of the roles inside the organisation, including a short description of the key tasks for each, can be found in the table below. %don't ask me why using the ref ain't working 

\begin{table}[h]
   \begin{tabularx}{\textwidth}{ | p{6.9cm} | p{6.9cm} | }
   
   \hline
   \emph{Description} & \emph{Key Tasks} \\ 
   \hhline{==}
   
   \multicolumn{2}{ | c | }{\textbf{Agile Coach}}
   
   \\ \hline
   
   \begin{itemize}[label={}, leftmargin=*, topsep=0pt, itemsep=0pt, partopsep=0pt]
     \item Coaches the organisation in the new ways of working by covering problematic and uncertain areas not handled by the existing organisational roles. Coaches the leadership team and management but also interacts with individual \acp{XFT} and other agilean roles when needed. After the settlement of new ways of working, the responsibilities are handed over to Section Managers. 
   \end{itemize} &
   
   \begin{itemize}[label={}, leftmargin=*, topsep=0pt, itemsep=0pt, partopsep=0pt]
     \item Drive agile and lean improvements in the organisation.
     \item Give feedback on ways to improve working.
     \item Drive workshops and retrospectives related to agile and lean.
     \item Coach teams to improve and become high-performing.
     \item Investigate, find and propose methods to improve teams and organisation.
     \item Participate in meetings when applicable (for Leadership Team, XFT, Program, Community of Practice, etc.).
     \item Work as a bridge between organisations on items related to ways of working and agile and lean.
   \end{itemize}
   
   \\ \hline
   
   \end{tabularx}
\end{table}

\begin{table}[h]
   \begin{tabularx}{\textwidth}{ | p{6.9cm} | p{6.9cm} | }
   
   \hline
      
   \multicolumn{2}{ | c | }{\textbf{Feature Project Manager}} 
   
   \\ \hline
   
   \begin{itemize}[label={}, leftmargin=*, topsep=0pt, itemsep=0pt, partopsep=0pt]
     \item End-to-end responsible for features/other work items in case of them being too large and complex to be managed by a single \ac{OPO}. Handles related coordination and progress reporting. 
   \end{itemize} & 

   \begin{itemize}[label={}, leftmargin=*, topsep=0pt, itemsep=0pt, partopsep=0pt]
     \item Support \acp{OPO} and teams with planning and coordination (e.g. between OPOs, teams, standards, projects, PDUs etc).
     \item Provide time plans and status updates.
     \item Report the status and escalate issues when needed.
     \item Follow up and reporting.
     \item Represent their part of a complex feature on a bigger scale.
   \end{itemize} 
   
   \\ \hline
   
   \multicolumn{2}{ | c | }{\textbf{Operative Product Owner (OPO)}}
   
   \\ \hline
   
   \begin{itemize}[label={}, leftmargin=*, topsep=0pt, itemsep=0pt, partopsep=0pt]
     \item Acts as a customer on the site for 2-5 \acp{XFT}, shaping team's backlog. Follows the quality of the developed feature and makes sure its end value is understood by the \ac{XFT}. Involved in technical development aspects, such as integration risks and technical dependencies.
   \end{itemize} & 
   
   \begin{itemize}[label={}, leftmargin=*, topsep=0pt, itemsep=0pt, partopsep=0pt]
     \item Prioritize user stories across backlogs.
     \item Give feedback on end to end time plan.
     \item Handle teams' backlogs and know their status.
     \item Provide delivery time plan and input to check-lists.
     \item Facilitate cross \acp{XFT} learning by diversifying user stories or arranging meetings.
   \end{itemize} 
   
   \\ \hline
      
   \multicolumn{2}{ | c | }{\textbf{Product Guardian (PG)}} 
   
   \\ \hline
   
   \begin{itemize}[label={}, leftmargin=*, topsep=0pt, itemsep=0pt, partopsep=0pt]
     \item Secures product quality by ensuring unity of architecture and code structure within product/domain and alignment to software outside the domain. Knowledgeable within a specified domain of software and uses their skill to support \acp{XFT} and build up competence in the organisation. Actively cooperates with \acp{OPO} on product improvement items to be put into \ac{XFT}'s backlogs.
   \end{itemize} & 
   
   \begin{itemize}[label={}, leftmargin=*, topsep=0pt, itemsep=0pt, partopsep=0pt]
     \item Help in technical decisions related to a product/domain that goes in line with fulfilling the product vision and quality requirements.
     \item Have a vital few design rules for the product.
     \item Support the creation of definition of done for features affecting the product.
     \item Collect and prioritize product care and improvement items.
     \item Coach less experienced people when working with the product's code, documents and test.
   \end{itemize} 
   
   \\ \hline
   
   \end{tabularx}
\end{table}

\begin{table}[h]
   \begin{tabularx}{\textwidth}{ | p{6.9cm} | p{6.9cm} | }
   
   \hline
   
   \multicolumn{2}{ | c | }{\textbf{Program Manager}}
   
   \\ \hline
   
   \begin{itemize}[label={}, leftmargin=*, topsep=0pt, itemsep=0pt, partopsep=0pt]
     \item Manages the program backlog, which is focused around a group of requirements from the product line.
   \end{itemize} & 
   
   \begin{itemize}[label={}, leftmargin=*, topsep=0pt, itemsep=0pt, partopsep=0pt]
     \item Discuss requirements and their release with the \ac{APO}.
     \item Facilitate program meetings with \acp{OPO} where they pull items from the backlog.
     \item Appoint a Feature Project Manager when main requirements are too much to handle for \acp{OPO}.
   \end{itemize} 
   
   \\ \hline
   
   \multicolumn{2}{ | c | }{\textbf{Section Manager (SM)}}
   
   \\ \hline 
   
   \begin{itemize}[label={}, leftmargin=*, topsep=0pt, itemsep=0pt, partopsep=0pt]
     \item Combines legal personnel responsibility and support of the \acp{XFT} by removing impediments that the teams cannot handle themselves and helping out with competence planning.  
   \end{itemize} & 
   
   \begin{itemize}[label={}, leftmargin=*, topsep=0pt, itemsep=0pt, partopsep=0pt]
     \item Participate frequently in \ac{XFT}'s stand-ups, demos, backlog preparations.
     \item Give feedback to \acp{XFT} and Scrum Masters.
     \item Involve Scrum Masters in discussions about team set ups, recruitments and processes.
   \end{itemize} 
   
   \\ \hline
   
   \multicolumn{2}{ | c | }{\textbf{Team Coach}} 
   
   \\ \hline
   
   \begin{itemize}[label={}, leftmargin=*, topsep=0pt, itemsep=0pt, partopsep=0pt]
     \item With application of Scrum as an area of expertise, acts as an Agile Coach on a team level typically for 2 \acp{XFT} by e.g. facilitating workshops and introducing methods.
   \end{itemize} & 
   
   \begin{itemize}[label={}, leftmargin=*, topsep=0pt, itemsep=0pt, partopsep=0pt]
     \item Give feedback on ways to improve working.
     \item Drive workshops related to agile and lean questions.
     \item Coach teams to improve and become high-performing.
     \item Investigate, find and propose methods to improve teams and organisation.
     \item Participate in meetings when applicable (for Leadership Team, XFT, Program, Community of Practice, etc).
     \item Handle impediments that the teams can not handle themselves.
   \end{itemize} 
   
   \\ \hline
   
      \end{tabularx}
\end{table}

\begin{table}[!htb]
   \label{table:roledefinitions}
   \begin{tabularx}{\textwidth}{ | p{6.9cm} | p{6.9cm} | }
   
   \hline 
   
   \multicolumn{2}{ | c | }{\textbf{XFT Scrum Master}}
   
   \\ \hline 
   
   \begin{itemize}[label={}, leftmargin=*, topsep=0pt, itemsep=0pt, partopsep=0pt]
     \item Ensures adherence of the \ac{XFT}'s process to Scrum. Filters interactions from outside to their \acp{XFT} based on their helpfulness. Acts according to a traditional theory concept, encouraging the team to improve its development process.
   \end{itemize} & 
   
   \begin{itemize}[label={}, leftmargin=*, topsep=0pt, itemsep=0pt, partopsep=0pt]
     \item Communicate visions, goals and product backlog items to the \ac{XFT} and assure efficient backlog management techniques.
     \item Coach the \ac{XFT} into self-organisation and cross-functionality.
     \item Lead and coach the team in its Scrum adoption.
     \item Work with other Scrum Masters to increase the effectiveness of the application of Scrum in the organisation.
   \end{itemize} 
   
   \\ \hline

  \end{tabularx}
  \caption{Role descriptions}
\end{table}
