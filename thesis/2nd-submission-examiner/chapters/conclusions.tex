\chapter{Conclusions}
\label{chap:conclusions}

Agile methodologies have recently been applied within companies of bigger sizes than its methodologies were originally designed for \citep{kniberg2012agilespotify, benefield2008agileenterprise}. Resulting from this any case study in the area of large scale agile eventually contributes to the existing body of knowledge. This thesis, through two weekly rounds of daily surveys and following semi-structured interviews, focused on the belated integration difficulties related to information and communication flow within the organisation and their implicative affect on the working environment of the \acp{XFT}.

The discovered information challenges, which are the subject of the \emph{RQ1}, include information evaluation, problematic information sharing and gathering due to unknown sources and receivers, information overflow, and information distortion along with the timing issues of the information's travel. Through numerous improvements among which are information filtering, information persisting guidelines and accessible Intranet, established information paths, homogeneous knowledge distribution and status information visibility benefits can be obtained.

The aim of the \emph{RQ1} was to also discover the challenges related to communication within an organisation that has transferred to agile methodologies. The most dominant findings on the topic include communication bottlenecks, communication islands, differences in perspectives of communicating parties and geographical distribution. These call for improvements in delegation abilities for the \acp{XFT}, tight integration between parts of the organisation and technical equipment support. Working on improvements yields benefits for the agile organisation, which include reduced misunderstandings, transparency, and natural communication.

The study concludes, that seemingly secondary to software development, areas of communication and information flow within the organisation are the integral aspects to consider within the adoption and emergence of a new way of working. In an organisation faced with the mentioned challenges, adjusting the level of organisational transparency and the related trade-offs of \acp{XFT} empowerment and workflow is a step to influence the productivity of development teams as subject to \emph{RQ2}. It is characterised by various aspects, among which are dependencies, unplanned work, unknown domain, external influences, technical environment and product vision.

Finally, heat maps and social networks have been demonstrated as powerful instruments for visualising data on communication (\textit{RQ3}). While the heat maps allow for illustrating the differences between the communications of various natures and spotting the focus points of intense collaborations, social networks give a structured overview of the communication paths. Both visualisation instruments have been used to underpin the findings connected to differences in communication intensities and their relation to productivity determinants. However, heat maps and social networks have also showed to be misleading if put out of context and analysed independently.

\section{Implications for Researchers}

Despite having only one case, the study opens avenues for future research.
As a consequence of research on agile software development from a perspective of communication and information being sparse \citep{pikkarainen2008impactagilecommunication}, this study intended to impart a structured understanding surrounding it. Just as \citet{badampudi2013proddelay} unfold potential productivity delays, the study attempts to integrate independently discovered determinants into a bigger picture of forces and balances.

This comprehension calls for: \emph{(i)} relating causes of productivity determinants with different trade-off positioning. One approach would entail performing a two-time study with a single organisation changing a single trade-off positioning before and after a transformation while looking at resulting productivity determinants. In another approach a multi-site study could be conducted with different organisations, which are positioned on different areas of the trade-offs, to compare the productivity determinants; \emph{(ii)} conducting surveys for constructing heat maps and social networks over a longer time-span with more \acp{XFT} using an automated data collection instrument to reduce the required effort. Such investigation holds potential to bring forth profound and more diverse findings around the teams' communication behaviours.

The study demonstrated the application of heat maps and social networks for visualisation of data of quantitative nature with a special focus on communication intensities. The instrument could be applied to investigate the issues of the similar nature, identify the flaws of its design and refine it with a purpose or generalisation for future studies.

\section{Implications for Practitioners}

Scaling agile within a large context entails various difficulties for practitioners. Even though agile's methodologies tend to have the general ability to scale or can be extended, emerging challenges of information and communication shall not be forgotten when trying to optimise adoption of agile.  

The outlined findings emphasize the importance of reflecting on trade-offs around the organisation, \ac{XFT} empowerment and the \ac{XFT}'s workflow. A then deeper analysis of existing communication and information challenges yields benefits through specific improvements. This eventually leads to an understood environment with specific productivity determinants which themselves can be influenced by adjusting the mentioned trade-offs and improving on information and communication's challenges.

It is important to note that qualitative investigations by capturing needed knowledge around the communication's and information's status quo are vital to lay a foundation. The proposed analysis should be applied in practice, reflected upon and eventually extended. The utilisation would give insights on how productivity determinants are affected by mitigated communication and information challenges within an environment defined by the mentioned trade-offs. Lastly, using heat maps and social networks for continuous real-time feedback within an organisation may reveal promising insights into an organisation's dynamics and associated issues.

The ability to incorporate qualities specific to any organisation's own nature allows the discussed framework to be applied outside the analysed case, ultimately contributing to a more successful application of agile without suggesting or prescribing concrete methods or practices.