\section{Agile and Organisational Culture}

In many ways agile embodies essential changes to management and organisational culture feeding back into the demeanour of individuals and the company's values~\citep{nerur2005migratingagile}.
The transformation moves towards an environment in which decisions are often made collaboratively and responsibilities are shared. This in turn also changes the management's range of duties from controlling most processes to fostering and mitigating obstacles of development teams~\citep{leffingwell2007scalelargecorps}. In addition,~\citet{2013schweigertamm} stress agile's significant influence by rather theoretical organisational culture aspects such as power distance and uncertainty avoidance.
In this context,~\citet{nerur2005migratingagile} argue that a transition from development- over process- to a people centric environment has taken place. More specifically, \citet{cockburn2001agilepeoplefactor} hold the belief that any project is shaped by personalities, different skills and the working environment paired with the organisational culture. At the same time, organisational culture is formed by focusing on individuals by empowerment and promotion of mutual trust to minimize the need for processes~\citep{cockburn2001agilepeoplefactor}. Ultimately, it helps to form an organic organisation by moving away from its mechanic counterpart to embrace social action on a corporate level~\citep{nerur2005migratingagile}.

Culture left alone is also often perceived with a predefined and static degree of fitness in terms of its agility~\citep{ivari2011orgagile}. The academia has followed transformations from plan-driven to agile software development in organisations of various size to investigate possible obstacles and encountered issues~\citep{laanti2011nokia}. Here a lack of profound body of knowledge in the area of agile adoption in large organisations has been pointed out especially as new practices do not tend to be equally appreciated even among employees within a single level of the organisation. In this regard~\citet{ivari2011orgagile} abstract further and define the Competing Values Model (CVM) of organisational culture to propose a number of hypothesis to mostly relate and judge hierarchical structures with agility. Anyhow, open ends remain in relation to an emerging perspective towards culture in which practices and beliefs are connected and ever changing. In this context,~\citet{gallivan2005persculture} pay significance to the fact that organisational structures and cultures have mostly been analysed separately and calls for an integrated analysis. In particular as organisational structures tend to be fixed in theory~\citep{gallivan2005persculture} but the individuals within it are constantly reinterpreting a given organisational structure to make it more compatible to a changing internal and external environment~\citep{iivari2010usability}. This belief is shared by~\citet{robey2000learning} who point out that organisational learning as a constant persistence of learnings in an evolving environment is usually underestimated in industry.

~\citet{schneider2000regeineering} looks deeper into the compatibility of cultures, processes and working environments. He defines four cultures in collaboration, control, cultivation and competence among the axis of personal to impersonal and potentiality to actuality. Each culture is distinct in regard to its fitness towards certain methodologies. Yet the scale is not to be confused with general maturity as it is just meant to raise awareness in respect to a potential need for change to apply certain practices being part of agile~\citep{schneider2000regeineering}.

The interplay of a pre-existent organisational culture and the people centrality of it is crucial to the success of applying agile as it demands constant change and revaluations of its context.