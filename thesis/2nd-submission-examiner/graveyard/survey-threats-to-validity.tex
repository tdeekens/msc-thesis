\subsection{Communication heat maps data collection and analysis}

\begin{description}
    \item[Representativeness] the respondent sample might not be sufficient to demonstrate and infer the communication patterns within the whole organization.
    \item[Maturation] categorization of intensity and nature of communication by the respondents might change over time due to the perception shift towards the answer options available.
    \item[Instrumentation] using the daily surveys to determine communication patterns is highly dependent on respondents' answers. Some of the actual conversations might be forgotten and unintentionally not included in the response form. Observations by a independent party could solve the issue but were disregarded due to unacceptably high demand of time investment.
    \item[Classification of the communication] nature items was based on previous studies at Ericsson along with information obtained from two members of the \ac{XFT}s. Therefore the list is rather subjective and possibly not complete with respect to existing issues. This is addressed by including "other" option. 
    \item[Intensity of communication] is assessed via a scale restricting the respondents to use only a few options. Such categorization limits the precision of communication intensity illustration but encourages the respondent to pick a certain category avoiding the possibility of obtaining evenly distributed communications over the day what could have happened in the case of e.g. slider type of questions where it is possible to continuously rate items.
    \item[Evaluation apprehension] being afraid of evaluation by their nature some people may answer the survey trying to look "better", e.g. not wanting their communications being discovered as they can be perceived as not work-related and distracting and even show incompetence of the person, or, on the other hand, trying to seem more communicative and approachable to show the value of their expertise for the organization.
\end{description}