\subsection{Transition Process}

Dissatisfied with performance, around 2008 Ericsson started a transformation from the waterfall based development towards a more agile approach, following small incremental and discontinuous transformation steps. Rather than only changing the lower level coordination of development teams, it was decided to change the organizational structure along the way. A matrix-like organizational structure was replaced with hierarchical one with cross-functional teams at its lower level trying to embrace agile software development: a structure not necessarily prescribed by agile but motivated by Ericsson's scale.
The strictly hierarchical structure causes a great number of connections, clear responsibilities and therefore delegation and potential queuing delays. As performance fluctuations and discrepancies became visible over the course of the transformation, it seems that especially boundaries and connection within the structure may hinder taking full advantage of agile software development.


\section{Agile at Ericsson}

Increasing productivity remains the main force behind the application of agile methodologies in industry. Nevertheless,~\citet{badampudi2013proddelay} points out that most companies' adapting agile do not manage to strictly follow all of its main ideas. Adjustments are made to integrate agile with the large scale and existing processes, with both positive and negative impacts upon productivity.

\subsection{History of Transformation}

Dissatisfied with performance, around 2008 Ericsson started a transformation from the waterfall based development towards a more agile approach, following small incremental and discontinuous transformation steps. Rather than only changing the lower level coordination of development teams, it was decided to change the organizational structure along the way. A matrix-like organizational structure was replaced with hierarchical one with cross-functional teams at its lower level trying to embrace agile software development: a structure not necessarily prescribed by agile but motivated by Ericsson's scale. The strictly hierarchical structure causes a great number of connections, clear responsibilities and therefore delegation and potential queuing delays. To address this Ericsson integrated parts of agile's methodologies into their existing organisational structure while adding variations of methodologies where needed. This includes a partially deploying methodologies, defining custom roles and responsibilities and adding an integration layer for the organisational structure.

\begin{comment}

# Prehistory
- Matrix organisation
- 

# Reasoning
- Enhance productivity
- Mention trade-off and interrelation between planning and and short-term view in agile
- 

# Expectations
- 

# Current challenges
- Information sink (responsibility pressed downwards into ~\ac{XFT}s)
- Shared understanding and communication between teams
- Understanding the overall vision and direction
- Integrating works and making process, technical foundation improvements available to all
- 

\end{comment}

\subsection{Organizational Structure at Ericsson}
\label{Chap:OrgStrucEricsson}

PDU LMR/Configuration and O\&M, located on two sites, Lindholmen and Kista (Sweden), is one of the Ericsson AB organisations with around XXX employees. 

Ericsson is a large multi-site corporation with over 100 000 employees. The scope of this study is limited  Ericsson's single organization, PDU LMR/Configuration and O\&M, located in Lindholmen, Sweden.
On the lowest level of the organizational structure and in the epicentre of the activities surrounding implementation work within the company reside the ~\ac{XFT}s. An~\ac{XFT} is a self-sufficient unit which has all the necessary competencies for a feature delivery at their disposal.~\ac{XFT}s at Ericsson generally consist of 5 to 9 people. To handle the size and complexity of the products the teams work with, they are appointed with a single ~\ac{PG}, who is not a part of a single~\ac{XFT} but works closely with several. Each team belongs to a section, where they have a ~\ac{TC} (also a ~\ac{SM}), who is appointed to 2~\ac{XFT}s as their supervisor. Sections themselves belong to departments, which in turn are a part of a sector, each of them having a respective manager. A structure above the sector in the hierarchy is called a~\ac{PDU}.\\
The start of the transformation towards agile development process caused introduction of a new product owner community to the existing structure. Given the company's scale, the traditional role of a~\ac{PO} had to be divided in to the areas of responsibility. Thus, new roles of ~\ac{TPO},~\ac{APO} and~\ac{OPO} were introduced with TPO being in contact with the customer and~\ac{APO}s, who in turn work closely together with 1-5~\ac{OPO}s each.~\ac{OPO}s work with several~\ac{XFT}s at a time, depending on which features from the program backlog they have been assigned to. In case of a feature being too large and complex,~\ac{FPjM} acts as an intermediary between the~\ac{PgM} and~\ac{OPO}s.

\begin{figure}[h!]
  \centering
  \includegraphics[width=0.75\textwidth]{figures/onion_001.png}
  \caption{Layers of roles interactions}
  \label{onion}
\end{figure}

Figure~\ref{onion} demonstrates the division of collaboration between the roles into several layers. An~\ac{XFT} is always in the closest and intermediate contact and cooperation with their~\ac{PG},~\ac{OPO} and~\ac{SM}/\ac{TC}. \todo{Mention TM and their role} The next layer is comprised of those roles who have a frequent contact to the~\ac{XFT} and their environment while the degree of this contact is significantly lower than in the first layer. \todo{Mention PM and how it differs between teams} The further the layer is from the~\ac{XFT} the least communication is implied.

\subsection{Role Descriptions \& Definitions}

The setting of this case study is peculiar to a single organisation and thus calls for additional description. With the elements of the pre-transformation organisational structure still in place and introduction of the new units called for by Scrums' methods, boarders between the responsibilities of different roles might at times seem unclear.

\begin{table}[h]
   \begin{tabularx}{\textwidth}{ | p{6.9cm} | p{6.9cm} | }
   
   \hline
   \emph{Description} & \emph{Key Tasks} \\ 
   \hhline{==}
   
   \multicolumn{2}{ | c | }{\textbf{Agilean Coach}}
   
   \\ \hline
   
   \begin{itemize}[label={}, leftmargin=*, topsep=0pt, itemsep=0pt, partopsep=0pt]
     \item Coaches the organisation in the new ways of working by covering problematic and uncertain areas not handled by the existing organisational roles. Mostly coaches the leadership team and management but also interacts with individual XFTs and other agilean roles when needed. After the settlement of new ways of working, the responsibilities are handed over to Section Managers. 
   \end{itemize} &
   
   \begin{itemize}[label={}, leftmargin=*, topsep=0pt, itemsep=0pt, partopsep=0pt]
     \item Drive agile and lean improvements in the organization
     \item Give feedback on ways to improve working
     \item Drive workshops and retrospectives related to agile and lean
     \item Coach teams to improve and become high-performing
     \item Investigate, find and propose methods to improve teams and organization
     \item Participate in meetings when applicable (for LT, XFT, Program, CoPs, etc.)
     \item Work as a bridge between organizations on items related to ways of working and agile and lean  
   \end{itemize}
   
   \\ \hline
   
   \multicolumn{2}{ | c | }{\textbf{Team Coach}} 
   
   \\ \hline
   
   \begin{itemize}[label={}, leftmargin=*, topsep=0pt, itemsep=0pt, partopsep=0pt]
     \item With application of Scrum as an area of expertise, acts as an Agilean Coach on a team level for 1-2 XFTs by e.g. facilitating workshops and introducing methods.
   \end{itemize} & 
   
   \begin{itemize}[label={}, leftmargin=*, topsep=0pt, itemsep=0pt, partopsep=0pt]
     \item Give feedback on ways to improve working
     \item Drive workshops related to agile and lean questions
     \item Coach teams to improve and become high-performing
     \item Investigate, find and propose methods to improve teams and organization
     \item Participate in meetings when applicable (for LT, XFT, Program, CoPs, etc)
     \item Handle impediments that the teams can not handle themselves 
   \end{itemize} 
   
   \\ \hline

   
   \end{tabularx}
\end{table}

\begin{table}[h]
   \begin{tabularx}{\textwidth}{ | p{6.9cm} | p{6.9cm} | }
   
   \hline
   
   \multicolumn{2}{ | c | }{\textbf{Feature PM}} 
   
   \\ \hline
   
   \begin{itemize}[label={}, leftmargin=*, topsep=0pt, itemsep=0pt, partopsep=0pt]
     \item End-to-end responsible for the features/other work items in case of them being too large and complex to be managed by a single OPO. Handles related coordination and progress reporting. 
   \end{itemize} & 

   \begin{itemize}[label={}, leftmargin=*, topsep=0pt, itemsep=0pt, partopsep=0pt]
     \item Support OPOs and teams with planning and coordination (e.g. between OPOs, teams, standards, projects, PDUs etc)
     \item Time plans and status updates
     \item Report the status and escalate issues when needed
     \item Follow up and reporting
     \item Represent their part of a complex feature on a bigger scale 
   \end{itemize} 
   
   \\ \hline
   
   \hline
   \multicolumn{2}{ | c | }{\textbf{OPO}}
   
   \\ \hline
   
   \begin{itemize}[label={}, leftmargin=*, topsep=0pt, itemsep=0pt, partopsep=0pt]
     \item Acts as a customer on the site for 2-5 XFTs, shaping team's backlog. Follows the quality of the developed feature and makes sure its end value is understood by the XFT. Involved in technical development aspects, such as integration risks and technical dependencies.
   \end{itemize} & 
   
   \begin{itemize}[label={}, leftmargin=*, topsep=0pt, itemsep=0pt, partopsep=0pt]
     \item Prioritize user stories across backlogs
     \item Give feedback on end to end time plan
     \item Handle teams' backlogs and know their status
     \item Provide delivery time plan and input to check-lists
     \item Facilitates cross XFTs learning by diversifying user stories or arranging meetings 
   \end{itemize} 
   
   \\ \hline

   
   \end{tabularx}
\end{table}

\begin{table}[h]
   \begin{tabularx}{\textwidth}{ | p{6.9cm} | p{6.9cm} | }
   
   \hline
   
   \multicolumn{2}{ | c | }{\textbf{PG}} 
   
   \\ \hline
   
   \begin{itemize}[label={}, leftmargin=*, topsep=0pt, itemsep=0pt, partopsep=0pt]
     \item Secures product quality by ensuring unity of architecture and code structure within product/domain and alignment to software outside the domain. Knowledgeable within a specified domain of software and uses their skill to support XFTs and build up competence in the organization. Actively cooperates with OPOs on product improvement items to be put into XFT's backlogs.
   \end{itemize} & 
   
   \begin{itemize}[label={}, leftmargin=*, topsep=0pt, itemsep=0pt, partopsep=0pt]
     \item Help in technical decisions related to a product/domain that goes in line with fulfilling the product vision and quality requirements
     \item Have a vital few design rules for the product
     \item Support the creation of definition of done for features affecting the product
     \item Collect and prioritize product care and improvement items
     \item Coach less experienced people when working with the product's code, documents and test 
   \end{itemize} 
   
   \\ \hline

   \multicolumn{2}{ | c | }{\textbf{Scrum M}}
   
   \\ \hline 
   
   \begin{itemize}[label={}, leftmargin=*, topsep=0pt, itemsep=0pt, partopsep=0pt]
     \item Ensures adherence of the XFT's process to Scrum. Filters interactions from outside to their XFTs based on their helpfulness. Acts according to a traditional theory concept, encouraging the team to improve its development process.
   \end{itemize} & 
   
   \begin{itemize}[label={}, leftmargin=*, topsep=0pt, itemsep=0pt, partopsep=0pt]
     \item Communicate visions, goals and product backlog items to the XFT and assure efficient backlog management techniques
     \item Coach the XFT into self-organization and cross-functionality
     \item Lead and coach the organization in its Scrum adoption
     \item Work with other Scrum Masters to increase the effectiveness of the application of Scrum in the organization 
   \end{itemize} 
   
   \\ \hline

   \multicolumn{2}{ | c | }{\textbf{SM}}
   
   \\ \hline 
   
   \begin{itemize}[label={}, leftmargin=*, topsep=0pt, itemsep=0pt, partopsep=0pt]
     \item Combines legal personnel responsibility and support of the XFTs by removing impediments that the teams cannot handle themselves and helping out with competence planning.  
   \end{itemize} & 
   
   \begin{itemize}[label={}, leftmargin=*, topsep=0pt, itemsep=0pt, partopsep=0pt]
     \item Participate frequently in XFT's stand-ups, demos, backlog preparations
     \item Give feedback to the XFT and Scrum Master after Scrum meetings
     \item Involve Scrum Masters in discussions about team setups, recruitments and processes 
   \end{itemize} 
   
   \\ \hline

   \multicolumn{2}{ | c | }{\textbf{PgM}}
   
   \\ \hline
   
   \begin{itemize}[label={}, leftmargin=*, topsep=0pt, itemsep=0pt, partopsep=0pt]
     \item Also doing something.
   \end{itemize} & 
   
   \begin{itemize}[label={}, leftmargin=*, topsep=0pt, itemsep=0pt, partopsep=0pt]
     \item many many many tasks 
   \end{itemize} 
   
   \\ \hline

  \end{tabularx}
  \caption{Roles descriptions}
\end{table}

\subsection{Development Process}

Due to its scale Ericsson follows the interpretation of Scrum where the traditional, "textbook", concepts of the framework are being accordingly refined to satisfy organizational needs. Apart from the differences in roles  mentioned in~\nameref{Chap:OrgStrucEricsson}, the peculiarities of the process related to events and artefacts need to be elaborated upon.\\
The sprint planning and retrospective meetings are held not within single~\ac{XFT}s but rather within an area of responsibility of an~\ac{OPO}. The~\ac{OPO} drives the meeting where the main objective is communication of work with the teams informing the others about it and getting feedback. Each team's representative (usually a Scrum Master) reports on the deliverables of the sprint, encountered problems and possible discrepancies that might arise from the delivered components. The meetings are open: everybody can ask for specific information on deliverables and raise concerns or issues. After summing up the previous sprint, teams proceed to the sprint planning. To reduce its duration and make it less tiresome for the participants, teams hold the planning meeting within themselves one day in advance where they discuss the stories from the backlog and assign their points.\\
A team's backlog is managed by their \ac{OPO}, who in collaboration with a~\ac{SM} pick items for the next sprint. User stories are being broken down into tasks of a reasonably small and independent nature and supplemented with a definition of done. The team visualizes their work using a Scrum board which appearance may vary from team to team but at all times follows the progress of a task being in the waiting list, worked on, finished and done. The sprint progress is observed by means of burn down charts which may also differ slightly in format.