Heat maps and social networks are powerful instruments for visualising data on communication. While the heat maps allow for illustrating the differences between the communications of various natures and spotting the focus points of intense collaborations, social networks give a structured overview of the communication paths. However, they might also be misleading if put out of context and analysed independently, as has been demonstrated with the case of communication between an XFT and a \quotes{Department (acting Section) Manager}.

The purpose of this study is to extend the knowledge of applying agile at scale with respect to information and communication flow and their effect on productivity determinants related to the empowerment of \acp{XFT}. Relating to agile, academia has investigated communication within~\acp{XFT}~\citep{stray2012investigatingdailyteam, kauffeld2011meetingsmatter, pikkarainen2008impactagilecommunication} and the productivity of agile methodologies in general~\citep{badampudi2013proddelay, leffingwell2007scalelargecorps}. Furthermore, communication has been quite widely discussed in the field of~\ac{DSD}~\citep{curtis1988fieldstudysoftwaredesign, kraut1995coordinationinsd, kraut1995coordinationinsd, rbs2012softwarearchitecture} but has paid little focus on its relation and impacts of agile's methodologies.
Hence, results of the thesis complement the existing knowledge about difficulties accompanying agile integration in large organisations with communication and information associated challenges. 