\subsection{Organizational structures}

This subsection takes a broader view towards organisational structures, relates them to social systems and agile software development in particular. The closer look at characteristics, structures and types points out relevant impacts regarding agile software development. 

It is important to notice that from a theoretic perspective an organization at large is perceived as a social system. Social systems themselves can be categorised in accordance to their criterion for demarcation. The different manifestations of demarcations lead to form a shared identity and alignment towards common goals. Organizations as a social system most fundamentally delineate themselves in regards to its constitution and rights of authorising their members. This usually forms a tight bond and prescribes processes and interactions. Groups on the other side of the spectrum solely define themselves by their interaction frequency. Coalitions on the other hand are formed in connection to actions towards third parties. The notion of a companionship takes this approach further by members accepting limitations of their behaviours and allowed actions to achieve a common goal collectively.

In addition, organisational structures are aligned to the company's goals and strategies. They therefore also vary depending on their level of formality and should follow the company's strategy~\citep{chandler1962strategystructure}. Organisations are also often categorised according to their division of sectors. The most commonly used types are matrix, divisional and functional. 

\begin{description}
  \item[Functional structures] are often perceived as the most
intuitive and widely used structure as cohesiveness is created by types of work. Especially multi-national corporations tend to endorse functional structures as they allow for a seemingly easy geographical distribution of a company's sections such as production, management and marketing. Still, structuring an organization into strictly severed parts embodies known issues such as a lack of sufficient communication and connections leading to potentially low coherence overall. Generally divisions will achieve a high degree of specialization and expert knowledge but tend to fail at self-managing themselves requiring higher levels of management for coordination and integration~\citep{price2007hrm}.
  \item[Divisional structures]
segregate units based on single products or complete product ranges, allowing for sections to gain comprehensive knowledge within their product range. Levels of responsibility transferred to a division differ corresponding to the degree of a business level oversight. In any case the business layer will still have to undertake the task of aligning all units and delegating broad instructions. Potential drawbacks can manifest themselves in insufficient communication between divisions again leading to low coherence and eventually diverging productivity levels. Mostly caused by units showing most commitment towards their product while being less concerned with the business perspectives on a higher level or other compartments~\citep{price2007hrm}.
  \item[Matrix structures]
combine functional and divisional structures adding another layer coordinating product units. The matrix originates from each functional area of a division being coordinated by two superior entities. For one from a unit overseeing the functional divisions (e.g. the research group) for another from the responsible division manager. The intent is to faster distribute knowledge horizontally among functional sectors while taking in advantages from both functional and divisional structures. Anyhow, the additional coordination blurs the schedule of responsibilities as teams are influenced from two managerial entities which also overall increases the degree of personal needed for management~\citep{galbraith2008matrix}.
\end{description}

% Old shorter version

\subsection{Organisational Structures}

Different organisations employ different structures to coordinate work and employees. The choice of a specific structure depends on the organisation's context and needs~\citep{baligh2010orgs}.

It is important to notice that from a theoretic perspective an organisation at large is perceived as a social system~\citep{takahara2003org}. Social systems themselves can be categorised in accordance to their criterion of demarcation~\citep{takahara2003org}. The different manifestations of demarcation lead to form a shared identity and alignment towards common goals. Organisations as social systems most fundamentally delineate themselves in regards to their constitution and rights of authorising their members~\citep{baligh2010orgs}. This usually forms a tight bond and prescribes processes and interactions. In contrast, groups in general solely define themselves by their interaction frequency. Coalitions on the other hand are formed in connection to actions towards third parties. The notion of a companionship takes this approach further by members accepting limitations of their behaviours and allowed actions to achieve a common goal collectively. In addition, organisational structures are aligned to the company's goals and strategies. They therefore also vary depending on their level of formality and should follow the company's strategy~\citep{chandler1962strategystructure}. 

Organisations are also often categorised according to their division of sectors. \emph{Functional structures} divide units by types of work achieving a high degree of specialisation and expert knowledge but tend to fail at self-managing themselves requiring higher levels of management for coordination and integration~\citep{price2007hrm}. \emph{Divisional structures}
segregate units based on single products or complete product ranges, allowing for sections to gain comprehensive knowledge within their product range. Potential drawbacks can manifest themselves in insufficient communication between divisions again leading to low coherence and eventually declining productivity level~\citep{price2007hrm}. \emph{Matrix structures}
combine functional and divisional structures adding another layer coordinating product units. The matrix originates from each functional area of a division being coordinated by two superior entities. The intent is to faster distribute knowledge horizontally among functional sectors while taking in advantages from both functional and divisional structures~\citep{galbraith2008matrix}. Finally, {hierarchical structures} add a chain of command and show superior and subordinate units or roles~\citep{healeyprojectmanagement}. They can be added to \emph{functional} and \emph{divisional} structures at various levels of detail.